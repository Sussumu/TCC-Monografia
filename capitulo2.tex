\chapter{OpenGL}
Ao longo do final do século XX, a computação gráfica ganhou grande importância na vida de pessoas comuns. Tal tecnologia se tornou presente não apenas em usos militares como SAGE (\textit{Semi-Automatic Ground Environment}), sistema que convertia dados de radares em imagens computadorizadas \cite{sage} (MACHOVER, 1978) na década de 50, como também em filmes como Toy Story, primeiro longa-metragem completamente computadorizado \cite{toystory} (GUHA, 2010).

Na década de 80, boa parte do trabalho gráfico era produzido em \textit{workstations}, que utilizavam a APIs com diferentes implementações, dificultando a produção de programas multiplataforma. Em 1982, a SGI (Silicon Graphics, Inc) começou o desenvolvimento de sua API proprietária IRIS GL, atingindo alta popularidade \cite{irisgl} (MARTZ, 2006). A pressão por uma API aberta e unificada aumentou por parte dos desenvolvedores de software, culminando em 1991 com a formulação do OpenGL ARB (\textit{Architecture Review Board}), consórcio de empresas para regulamentar o projeto, e o lançamento da versão 1.0 no ano seguinte.