\documentclass[brazil,ruledheader]{abntifes}
%\documentclass[brazil,twoside,ruledheader]{abntifes}
\usepackage[T1]{fontenc} 
%\usepackage[latin1]{inputenc}
\usepackage[utf8]{inputenc}
\usepackage[brazil]{babel}
\usepackage[]{algorithm2e}
\usepackage{pslatex}
\usepackage{url}
%\usepackage{fancyhdr}
\usepackage{graphicx}
\usepackage{amsmath, amsthm, amssymb}
\usepackage{exercise}
\usepackage{makeidx}
\usepackage{setspace}
\usepackage{multicol}    
\usepackage{upquote}
\usepackage{graphicx}
\usepackage{float}
\usepackage{epigraph}

\usepackage{listings}

\lstset{numbers=left, stepnumber=5, firstnumber=1, numberstyle=\tiny, extendedchars=true, breaklines=true, frame=tb, basicstyle=\footnotesize, stringstyle=\ttfamily, showstringspaces=false }

%\makenomenclature

% Para listar programas em C# 
\lstdefinelanguage{cs}
  {morekeywords={abstract,event,new,struct,as,explicit,null,switch
		base,extern,object,this,bool,false,operator,throw,
		break,finally,out,true,byte,fixed,override,try,
		case,float,params,typeof,catch,for,private,uint,
		char,foreach,protected,ulong,checked,goto,public,unchecked,
		class,if,readonly,unsafe,const,implicit,ref,ushort,
		continue,in,return,using,decimal,int,sbyte,virtual,
		default,interface,sealed,volatile,delegate,internal,short,void,
		do,is,sizeof,while,double,lock,stackalloc,
		else,long,static,enum,namespace,string, },
	  sensitive=false,
	  morecomment=[l]{//},
	  morecomment=[s]{/*}{*/},
	  morestring=[b]",
}

\newcommand{\AUTOR}{Gabriel Sussumu Kato}
\newcommand{\SEGUNDOAUTOR}{}
\newcommand{\ORIENTADOR}{D.Sc. Sicilia Ferreira Ponce Pasini Judice}
\newcommand{\COORIENTADOR}{}
\newcommand{\TITULO}{INTEGRAÇÃO OPENGL E BULLET PHYSICS}
\newcommand{\CURSO}{Tecnólogo em Tecnologia da Informação e da Comunicação}
\newcommand{\GRAU}{Tecnólogo em Tecnologia da Informação e da Comunicação}
% \newcommand{\GRAU}{Tecnólogo em Análise e Desenvolvimento de Sistemas}
\newcommand{\INSTITUICAO}{Faculdade de Educação Tecnológica do Estado do Rio de Janeiro Faeterj/Petrópolis}
\newcommand{\ANO}{Junho, 2016}
\newcommand{\DATA}{Dia de Mês de Ano}
\newcommand{\LOCAL}{Petrópolis - RJ}
\newcommand{\epigrafe}{\vspace{1cm}{\raggedright\par\sffamily\slshape\par}}
\begin{document}

\autor{\AUTOR}
\titulo{\TITULO}
\orientador{\ORIENTADOR}
\coorientador{\COORIENTADOR}

\comentario{Trabalho de Conclusão de Curso apresentado à Coordenadoria do Curso de \CURSO\
	    da \INSTITUICAO , como requisito parcial para obtenção do título de \GRAU .}

\instituicao{\INSTITUICAO}
\curso{\CURSO}
\governo{Governo do Estado do Rio de Janeiro}
\secretaria{Secretaria de Estado de Ciência e Tecnologia}
\fundacao{Fundação de Apoio à Escola Técnica}
\cpti{Centro de Educação Profissional em Tecnologia da Informação}
\local{\LOCAL}
\data{\ANO}

\capa

\folhaderosto

% Ficha Catalográfica
%\begin{figure}
%\includegraphics[width=11cm]{FichaCatalografica.pdf}
%\end{figure}

% Folha de Aprovação
\newpage
\vfill 
\null
\begin{center}
{\Huge {\bfseries\itshape Folha de Aprovação}}\\[3cm]
\begin{espacoduplo}
Trabalho de Conclusão de Curso sob o título \textit{``\TITULO''},
defendida por \AUTOR\ e aprovada em \DATA, em \LOCAL, pela banca examinadora constituída pelos
professores: \setlength{\ABNTsignthickness}{0.4pt}
\end{espacoduplo}
\setlength{\ABNTsignthickness}{0.4pt}

% ou inserir a página assinada e escaneada aqui
%\begin{figure}
%\includegraphics[]{Fo	lhaAprovacao.pdf}
%\end{figure}


\assinatura{Prof. \ORIENTADOR\\ Orientador} 
\assinatura{Prof. Banca Interna \\ \INSTITUICAO} 
\assinatura{Prof. Banca Interna \\ \INSTITUICAO} 
%\assinatura{Prof. Banca Externa \\ Instituto do membro externo}

\end{center}


% Folha do Termo de Compromisso
\newpage

\vfill 
\null
\begin{center}
{\Huge {\bfseries\itshape Declaração de Autor}}\\[3cm]
\begin{espacoduplo}
Declaro, para fins de pesquisa acadêmica, didática e técnico-científica, que o presente Trabalho de Conclusão
de Curso pode ser parcial ou totalmente utilizado desde que se faça referência à fonte e aos autores.
\end{espacoduplo}
\setlength{\ABNTsignthickness}{0.4pt}
\assinatura{\AUTOR}
Petrópolis, em \DATA
\end{center}


%\chapter*{Dedicatória}


%\chapter*{Agradecimentos}

\vfill 
\null

%\begin{center}
%{\Huge {\bfseries\itshape Epígrafe}}\\[3cm]
%\vspace{15cm}
%\end{center}
%\begin{espacoduplo}
%\end{espacoduplo}
% Não é obrigatorio ter epigrafe
%\epigraph{"Frase de efeito"}{(Autor)}

\begin{resumo}
Este trabalho demonstra o processo de integração entre a API (\textit{Application Programming Interface}, ou Interface de Programação de Aplicativo) gráfica OpenGL e o motor de física \textit{Bullet Physics}. Nele, foi desenvolvida uma simples demonstração onde é possível observar diversos fenômenos físicos como gravidade, colisão, inércia e fricção. Tal integração foi feita visando tornar mais agradável o aprendizado de qualquer uma das tecnologias envolvidas. A aplicação foi feita utilizando a biblioteca SDL2 (\textit{Simple DirectMedia Layer}), para tratar a criação de janelas e entrada e saída de dados do usuário em múltiplas plataformas, como Windows, MacOS e Linux. Para a criação do ambiente gráfico, foi utilizada a API gráfica multiplataforma OpenGL, escrita na versão 3.3 ou superior, também conhecida como \textit{modern}, o que possibilita utilizar recursos não existentes em versões anteriores. O motor de física Bullet tem suporte a recursos comuns, como gravidade, colisão, corpos rígidos, entre outros. Todas as bibliotecas escolhidas são de código livre. A ferramenta escolhida para o desenvolvimento foi a IDE (\textit{Integrated Development Environment}, ou Ambiente de Desenvolvimento Integrado) Visual Studio Community 2015, para Windows, também gratuita, porém é possível utilizar qualquer outro compilador para obter o mesmo resultado.
\\\\\textbf{Palavras-chave:} API gráfica. Motor de física.
\end{resumo}

\begin{abstract}
This work shows the integration process between the graphical API (Application Programming Interface) OpenGL and the physics engine Bullet Physics. It was developed a simple demonstration where it can be observed some physical phenomena like gravity, collision, inertia and friction. Such integration was created in order to make more pleasant learning any of the involved technologies. The application was made using the library SDL2 (Simple Direct Media Layer), to manage windows creation and input and output user data on multiple platforms, like Windows, MacOS and Linux. For the creation of the graphical environment, it was used the multiplatform graphical API OpenGL, written in version 3.3 or superior, also referred to as modern, what makes possible to use features not available on previous versions. The physics engine Bullet has support to common features like gravity, collision, rigid bodies, among others. All chosen libraries are open source. The selected tool for development were the IDE (Integrated Development Environment) Visual Studio Community 2015, for Windows, also free, however it is possible to use any other compiler to achieve the same result.
\\\\\textbf{Key-words:} Graphical API. Physics engine.
\end{abstract}
\listoffigures

%Lista de abreviaturas

\tableofcontents{}


\include{introducao}
\include{capitulo1}
\chapter{OpenGL}
Ao longo do final do século XX, a computação gráfica ganhou grande importância na vida de pessoas comuns. Tal tecnologia se tornou presente não apenas em usos militares como SAGE (\textit{Semi-Automatic Ground Environment}), sistema que convertia dados de radares em imagens computadorizadas (MACHOVER, 1978) na década de 50, como também em filmes como Toy Story (1995) (figura \ref{fig:toy-story}), primeiro longa-metragem completamente computadorizado (GUHA, 2010).

\begin{figure}[h]
	\centering
	\includegraphics[scale=0.35]{imagens/toy-story.jpg}
	\caption{\small Filme Toy Story. (Fonte: Pixar, 2016)}
	\label{fig:toy-story}
\end{figure}

Na década de 80, boa parte do trabalho gráfico era produzido em \textit{workstations}, que utilizavam a APIs com diferentes implementações, dificultando a produção de programas multiplataforma. Em 1982, a SGI (Silicon Graphics, Inc) começou o desenvolvimento de sua API proprietária IRIS GL, atingindo alta popularidade (MARTZ, 2006). A pressão por uma API aberta e unificada aumentou por parte dos desenvolvedores de software, culminando em 1991 com a formulação do OpenGL ARB (\textit{Architecture Review Board}), consórcio de empresas para regulamentar o projeto, e o lançamento da versão 1.0 no ano seguinte.

Segundo Wright (2013), o OpenGL é uma camada de abstração entre o \textit{software} e o sistema gráfico, devendo permitir que a aplicação execute independente dos diferentes tipos de \textit{hardware}. Além disso, a API precisa operar em diferentes sistemas operacionais, arquiteturas e resoluções de tela, ao mesmo tempo que expõe as características de cada \textit{hardware} para que o programador faça o melhor uso.

Em 2004, com o lançamento da versão 2.0, foi lançada a OpenGL \textit{Shading Language},

Com o tempo, várias funções foram adicionadas a especificação, o que tornava difícil a compatibilidade entre todas elas. Por esse motivo, em 2008, o OpenGL ARB decidiu criar dois perfis: \textit{core} e \textit{compatibility}, separando padrões suportados por arquiteturas modernas e obsoletas, respectivamente.


%===================================================================================
%\backmatter
%===================================================================================

%\bibliography{monografia}{}
%\bibliographystyle{abnt-alf}

\begin{thebibliography}{1}
\makeatletter
\renewcommand\@biblabel[1]{}
\makeatother

\bibitem{sage}MACHOVER, Carl. \textit{A Brief, Personal History of Computer Graphics}. 1978. p.38. Disponível em: \url{https://www.computer.org/csdl/mags/co/1978/11/01646756.pdf}.
Acesso em: 07 de junho de 2016.

\bibitem{toystory}GUHA, Sumanta. \textit{Computer Graphics Through OpenGL: From Theory to Experiments, Second Edition}. 2014. A K Peters/CRC Press. p.8-9.

\bibitem{irisgl}MARTZ, Paul. \textit{OpenGL Distilled}. 2016. Addison-Wesley Professional. p.26-8.

\bibitem{glbible}WRIGHT, Richard. HAEMEL, Nicholas. SELLERS, Graham. \textit{OpenGL SuperBible: Comprehensive Tutorial and Reference, 6th Edition}. 2013. Addison-Wesley Professional. p.4-9.

\bibitem{glsl1}ROST, Randi. LICEA-KANE, Bill. \textit{OpenGL Shading Language}. 2009. Addison-Wesley Professional. p.26-8.

\bibitem{toystory}Pixar. 2016. \textit{Pixar Feature Films}. Disponível em: \url{http://www.pixar.com/features_films/TOY-STORY}.
Acesso em: 08 de junho de 2016.

\bibitem{shaders}BAILEY, Mike. CUNNINGHAM, Steve. \textit{Graphic Shaders: Theory and Practice, 2nd Edition}. 2012. CRC Press. p.20-1.

\bibitem{glsl2}OpenGL. 2016. \textit{OpenGL Shading Language Specification}. Disponível em: \url{https://www.opengl.org/documentation/glsl/}.
Acesso em: 13 de junho de 2016.

\bibitem{gl2khronos}Khronos Group. 2016. \textit{OpenGL ARB to Pass Control of OpenGL Specification to Khronos Group}. Disponível em: \url{https://www.khronos.org/news/press/opengl_arb_to_pass_control_of_opengl_specification_to_khronos_group}.
Acesso em: 13 de junho de 2016.

\bibitem{glsl2}Khronos Group. 2016. \textit{About The Khronos Group}. Disponível em: \url{https://www.khronos.org/about/}.
Acesso em: 13 de junho de 2016.


\end{thebibliography}


\anexo
\end{document}